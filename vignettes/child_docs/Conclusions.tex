\section{Conclusions} \label{conclusion_section}

In this article, we have presented a novel source attribution model which builds upon, and unites, the Hald and Modified Hald approaches. This model allows the data to inform type effect clustering 
using a Bayesian non-parametric model. This is a significant improvement over 
the previous attempts to improve model identifiability by reducing the effective number of parameters (fixing some source and type effects, or modelling the type effects as random using a 2 stage 
model).  Like the Modified Hald model, the new model incorporates uncertainty in the prevalence matrix into the model, however, it does this by fitting a fully joint model rather than a 2 step model. This 
has the advantage of allowing the human cases to influence the uncertainty in the source cases and preserves the restriction on the sum of the prevalences for each source. The \pkg{sourceR} package implements this flexible Bayesian non-parametric model to enable straightforward attribution of cases of zoonotic infection to putative sources of infection by epidemiologists and other scientists.