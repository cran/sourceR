\section{Introduction}

\subsection{Background}

%brief overview of the motivation and aim of this methodology.  Briefly review the outcomes of past work (Petra papers, more recent stuff).  Highlight the limitations of the previous approach.

Food-borne diseases are a major source of human morbidity and mortality world wide. In 2010, an estimated 600 million cases occurred globally, of which approximately 90\% were cause by food 
borne diarrhoeal disease pathogens \citep{HavEtAl2015}.  Identifying the source from which a food-borne disease is acquired, and the pathway by which it enters the food chain, is crucial for the 
identification and prioritization of food safety interventions.  Traditional approaches to source attribution include full risk assessments, analysis and extrapolation of surveillance or outbreak data, and
analytical epidemiological studies \citep{CrGrAn02}.  However, their results may be highly uncertain due to long and variable incubation times of food-borne diseases in the face of many and various 
exposures of an individual to potential sources.  Given this difficulty, quantitative methods using pathogen genotype frequency have become popular for identifying important sources of food-borne 
illness \citep{MulJonNob09}.  

For a given disease, quantitative source attribution relies on molecular typing data from pathogen genotypes isolated from human cases as well as from a number of putative sources of infection.   
For bacterial diseases, source samples are usually collected from food (such as raw chicken, beef etc) and environmental (such as water or faeces samples) sources, and tested for the presence of 
the zoonotic bacterium usually using polymerase chain reaction (PCR) methods. The bacterial samples are then categorised into subtypes using a genetic typing methodology.  Multilocus Sequence Typing (MLST) is commonly used because it is a relatively 
inexpensive, rapid, and unambiguous procedure for coarse characterisation of isolates of bacterial species. Here, genetic variations in small fragments of several house-keeping genes are assigned 
distinct allelic identifiers.  A sequence type is therefore defined as a unique combination of alleles at each gene locus \citep{DingColWar01}.  Being defined on conserved regions of the bacterial 
genome, the evolution of new MLST types is slow, enabling data collected over a period of months to be classed as cross-sectional, making it suitable for use in source attribution models.  Recent 
statistical approaches designed specifically to use MLST data are reviewed in Section \ref{sec:review}. 

Routine surveillance for food-borne pathogens is now commonplace in many countries and is performed by national authorities, for example FoodNet in the US \citep{Allos15042004}, the Danish 
Zoonosis Centre (\url{food.dtu.dk}), and the Ministry for Primary Industries in New Zealand (\url{foodsafety.govt.nz}).  However, despite this availability of data there are no implementations in standard statistical software 
 for source attribution modelling, with analyses being performed using a variety of \emph{ad hoc} methodologies.  Moreover, as the example of human \emph{Campylobacter jejuni} 
cases in New Zealand between 2005 and 2007 shows, current statistical source attribution models are subject to computational approximations and inherent identifiability problems.  

This paper presents an \code{R} package \pkg{sourceR} implementing a flexible Bayesian non-parametric model, designed for use by epidemiologists and other scientists to attribute cases of 
zoonotic infection to putative sources of infection.  The paper is structured as follows.  We first describe a motivating example in Section \ref{sec:motivation}, before briefly reviewing a set of related 
models that have been previous applied to this dataset in Section \ref{sec:review}, and for which our model represents a significant advance.  We describe our source attribution model in Section \ref{sec:model}, and demonstrate its utility through worked examples on simulated and real-world data in Sections \ref{sim_study_section} and \ref{real_data_case_study_section} respectively. 

