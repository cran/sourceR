\section{Motivating example}\label{sec:motivation}

\emph{Campylobacter} is the most commonly identified cause of food-borne bacterial gastro-enteritis in the developed world \citep{MilMan05} is estimated to be responsible for over 26\% of bacterial foodbourne illnesses world-wide \citep{HavEtAl2015}. % 95613970/(349405380+10342024)
%90\% of all reported cases of bacterial food poisoning world-wide \citep{Thorns2000}. 
In 2006, New Zealand had one of the highest incidences of campylobacteriosis in 
the developed world, with an annual incidence in excess of 400 cases per 100,000 people \citep{BakWilIkr06}.  A campaign to change poultry processing procedures, supported in part by results from 
quantitative source attribution methods, was successful in leading to a sharp decline in campylobacteriosis incidence after 2007 \citep{MulJonNob09}.  This example provides the dataset that motivates the construction of the \pkg{sourceR} package.  The dataset was first published in \cite{MuelColMid10}, with a detailed description of the data (and data collection methods) available in \cite{FreMar09} and \cite{FreMar13}.  

Briefly, our data consist of MLST-genotyped \emph{Campylobacter jejuni} isolates from both human cases of campylobacteriosis and potential food and environmental sources 
between 2005 and 2008 in the Manawatu region of New Zealand.  The human isolates were obtained from the local medical microbiology service (MedLab 
Central, Palmerston North), with isolates from food and environmental sources collected during a sample-based surveillance study.  Samples of beef and lamb were collected from local retail stores, water from popular local riverine swimming locations, and sheep and cattle faeces from farms within local river catchments.  These samples 
were then grouped into one of six sources: poultry supplier A, poultry supplier B, poultry supplier C, bovine (beef mince and liver, and cattle faecal samples), ovine (lamb mince and liver, and sheep 
faecal samples) and environmental (water samples).   

These data are included within \pkg{sourceR}, named \code{campy}, comprising a data frame of the number of positive isolates of each MLST 
type identified from humans and each potential source of infection.  We use this dataset as a source attribution case study in Section \ref{real_data_case_study_section}, comparing our results with previously 
published \emph{ad hoc} statistical approaches.