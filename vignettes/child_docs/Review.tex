\section{Review of models and notation}\label{sec:review}

This section briefly reviews the current source attribution models. Throughout, we adopt a convention where $i=1,\dots,n$ denotes a bacterial subtype, and $j=1,\dots,m$ denotes a putative source of 
infection.


\subsection{Dutch model}\label{sec:dutch}
The Dutch method \citep{Dutch1999} is one of the simplest models for source attribution. It compares the number of reported human cases caused by a particular bacterial subtype with the relative 
occurrence of that subtype in each source. The number of reported cases per subtype and reservoir is estimated by:
\begin{eqnarray}
\lambda_{ij}=\frac{r_{ij}}{\sum_j r_{ij}}y_i
\end{eqnarray}
where $r_{ij}$ is the relative occurrence of bacterial subtype $i$ in source $j$,
$y_i$ is the estimated number of human cases of type $i$ per year,
$\lambda_{ij}$ is the expected number of cases per year of type $i$ from source $j$. A summation across subtypes gives the total number of cases attributed to source $j$,
denoted by $\lambda_j$:
\begin{eqnarray}
\lambda_j=\sum_i \lambda_{ij} 
\end{eqnarray}
As the Dutch model has no inherent statistical noise model, confidence intervals for the estimated total attributed cases $\hat{\lambda_j}$ by bootstrap sampling over the dataset.

%%%%%%%%%%%%%%%%%%%%%%%%%%%%%%%%%%%%%%%%%%%%%%%%%%%%%%%%%%%%%%%%%%%%%%%%%%%%%%%%%%%%%%%%%%%%%%%%%
%%%%%%%%%%%%%%%%%%%%%%%%%%%%%%%%%%%%%%%%%%%%%%%%%%%%%%%%%%%%%%%%%%%%%%%%%%%%%%%%%%%%%%%%%%%%%%%%%
%%%%%%%%%%%%%%%%%%%%%%%%%%%%%%%%%%%%%%%%%%%%%%%%%%%%%%%%%%%%%%%%%%%%%%%%%%%%%%%%%%%%%%%%%%%%%%%%%

\subsection{Hald model}\label{sec:hald}
One of the first stochastic source attribution models was the Hald model \citep{HaldVosWed04} which was developed to model the source attribution of salmonellosis in Denmark. It extends the Dutch 
method by incorporating source and type effect parameters into the model, and assuming that the number of human cases are Poisson distributed conditional on the source typing data.  Source and 
type effect parameters are used to account for source- and type- specific influences on the rates. Type effects summarise the characteristics that determine a type's capacity to cause an infection 
(survivability, pathogenicity and virulence). Source effects account for the ability of a particular food source to act as a vehicle of infection. This is a significant advantage over the Dutch model as it is 
not plausible that type and source effects are equal for most zoonoses.  Inference is performed in a Bayesian framework allowing the model to explicitly include and quantify the uncertainty 
surrounding each of the parameters.

The number of human cases $y_i$ of isolate type $i=1,\dots,n$ is Poisson distributed such that 
\begin{eqnarray}
y_i & \sim & \mbox{Poisson} (\lambda_i) \\
\lambda_i & = & q_i\sum_{j=1}^{m} a_j c_j p_{ij}\label{eq:haldmodel}
\end{eqnarray}
where $p_{ij}=r_{ij}\times \pi_j$ is the absolute prevalence of each type in source $j$, $\pi_j$ is the prevalence of positive samples in source $j$, $c_j$ is the offset for 
the annual consumption of each food source $j$, $n_j$ is the total number of samples from each source, $r_{ij}=\frac{x_{ij}}{\sum_{i=1}^{I}x_{ij}}$ is the relative prevalence of each type in source $j$, 
$x_{ij}$ is the number of MLST positive samples for type $i$ in source $j$, $a_j$ is the $j^{th}$ source effect, and $q_i$ is the $i^{th}$ type effect. The rate of cases attributed to each source is given by $
\lambda_j=\sum_{i=1}^{n} a_j c_j p_{ij}$. 

Note that the prevalence $\pi_j$ is calculated by dividing the number of positive samples (using PCR to detect the presence of \emph{Campylobacter}) by the total number of samples for each source. Samples testing positive for \emph{Campylobacter} using PCR are MLST typed. It is possible for MLST typing to fail, hence, the number of positive samples for a given source (used in the prevalence calculation) can exceed the number of source samples used in the source data matrix ($x$).

This model is overparameterised because there are $m+n$ parameters (the source and type effects) but only $n$ independent observations (the observed human case totals $y_{i}$). Identifiability 
was obtained by assuming some source and type effects were equal. This was done by pooling the bacterial subtypes into groups (where types within the same group have the same type effect) and 
assuming the source effects were the same for Danish and imported pork. Although in some cases there may be some physical justification to set some parameters equal, it is not possible for all 
zoonoses.  Furthermore, the intensity of the source surveillance system in Denmark justified the use of point estimates of $p_{ij}$, rather than explicitly modelling the source sampling process.

%%%%%%%%%%%%%%%%%%%%%%%%%%%%%%%%%%%%%%%%%%%%%%%%%%%%%%%%%%%%%%%%%%%%%%%%%%%%%%%%%%%%%%%%%%%%%%%%%
%%%%%%%%%%%%%%%%%%%%%%%%%%%%%%%%%%%%%%%%%%%%%%%%%%%%%%%%%%%%%%%%%%%%%%%%%%%%%%%%%%%%%%%%%%%%%%%%%
%%%%%%%%%%%%%%%%%%%%%%%%%%%%%%%%%%%%%%%%%%%%%%%%%%%%%%%%%%%%%%%%%%%%%%%%%%%%%%%%%%%%%%%%%%%%%%%%%

%data on annual food consumption was not readily available,
\subsection{Modified Hald model}\label{sec:modifiedHald}
The Modified Hald model \citep{MulJonNob09} was developed  because the Hald model had some assumptions that were not suitable for modelling campylobacteriosis.  There was no evidence to 
justify \emph{a priori} fixing some source and type effects to be equal, and the source data came from a less intensive surveillance system with fewer source samples taken (suggesting it would be 
beneficial to introduce uncertainty into the source prevalence matrix).   For their application, they wished to include the environment as a potential source of infection.  Since it is not possible to 
quantify annual exposure to the environment, the annual consumption offset was removed from the model. 

The number of human cases $y_i$ of isolate type $i=1,\dots,n$ is again Poisson distributed with rate $\lambda_i$ for each type $i$ as in Equation \ref{eq:haldmodel}, omitting the annual consumption 
term $c_j$.  In contrast to the Hald model, identifiability of the model is ensured by treating $\bm{q}$ as a a log Normal$(0, \tau)$ distributed random effect.  However, a strong prior is needed on $\tau
$ to shrink $\bm{q}$ towards 0 sufficiently to avoid overfitting the model, the choice of which is arbitrary. 

In a further development, the modified Hald model introduces uncertainty into the relative prevalence matrix by modelling the source sampling process. The $p_{ij}$'s were first modelled in a separate 
Bayesian scheme, where independent symmetric Dirichlet priors were used to model columns of the $\bm{r}$ matrix, and a non-informative Beta distribution was used for the source prevalences:
\begin{eqnarray}
r_{\cdot j}\sim\textsf{Dirichlet}(\bm{1}) \; \forall \; j \label{eq:mhrij} \\
\pi_{j}\sim\textsf{Beta}(1,1) \; \forall \; j \label{eq:mhpij}
\end{eqnarray}

This model was fitted in WinBUGS using an approximate two stage process \citep{MulJonNob09}.  First, a posterior distribution was estimated for the absolute prevalence of source subtypes $\bm{p}
$, using the model specified in Equations \ref{eq:mhrij} and \ref{eq:mhpij}.  The marginal posterior for each element of $\bm{p}$ was then approximated by a Beta distribution
$$
p_{ij}\sim\textsf{Beta}(\alpha_{ij},\beta_{ij})
$$
using the method of moments for $\alpha_{ij}$ and $\beta_{ij}$.  which were included as independent priors in the Poisson model.  Due to convergence issues for very small $\alpha_{ij}$ values, $
\alpha_{ij}$ was limited to be at least 1 and the $\beta_{ij}$ parameter was adjusted accordingly \citep{FreMar09}.

Using independent Beta priors on each $p_{ij}$ removes the constraint that they sum to $\pi_{j}$ over each type $i$.  Thus, the absolute prevalence for source $j$ ($\sum_{i=1}^{I}p_{ij}$) is no longer 
constrained to be a probability. 



\subsection{Asymmetric Island model}

The Asymmetric Island Model \citep{WilGabLea08, iSource} takes a different approach to the models described above.  Here,  the evolutionary processes (mutation, migration and recombination) of the 
sequence types is modelled to probabilistically infer the source of each human infection. This means it requires genetic typing for all samples limiting the range of data that can be used with this 
model (for example, phenotypic typing cannot be used). The extra information in the genetic typing allows the model to attribute human cases not observed in any sources to a likely source of 
infection by looking at the genetic similarity of that type to other types that are observed in the sources; this is not possible with the Dutch, Hald or Modified Hald models. However, they are much 
simpler with fewer strong assumptions and work with a wider range of data than the Island model. 